%Kapitel 1

\section{Das erste Kapitel}

\begin{spacing}{1}
\begin{itemize}
	\item \info{Unterstreicht ein Wort und führt zur Sprechblase am Rand} Notizen zu Kapitel 1
\begin{itemize}
	
\item Hier steht Text\autocite[Vgl.][55\psqq]{Buch1}
\item \enquote{Hier steht noch [...] mehr Text}\autocites(Auslassung: \verfasser)[2]{Buch1}

\end{itemize}
\end{itemize}
\end{spacing}

Da wir Texte lieben und vor allem auch Fußnoten\autocite[1-X]{biblatex1} hier noch eine. 


Hier ist eine weitere\unsicher[fancyline, author=Korrektor]{Hellgrauer Pfeil direkt zu der Stelle, auf die sich die Anmerkung Bezieht} Anmerkung notwendig.

\blindtext

\verbessern{dieser Teil ist noch zu schlecht}Bla bla blaaaa
\newline

Was es hier noch Anzumerken gibt ist die schöne Lösung nicht von Forte oder \textit{piano} sprechen zu müssen, sondern die Standardsymbole \dynmark{f} und \dynmark{p} im Fließtext verwenden zu können.
Das sieht gleich viel professioneller aus, vor allem bei moderneren Angaben wie \dynmark{sfmf}.\info{Dieser Befehl löscht sich nicht automatisch wenn man die Option "print" aktiviert. Daher wird dieser recht obsolete Befehl beim nächsten Update vsl. Entfernt.}\perfekt

\blindtext

\begin{center}
\missingfigure[figwidth=7.2cm]{Hier ist Platz für ein Bild \ldots}
\end{center}

\blindtext

\textsl{Nun noch ein Beispiel für das Blockzitat:}

\begin{Blockzitat}
	\blindtext\autocite{Buch2}
\end{Blockzitat}
